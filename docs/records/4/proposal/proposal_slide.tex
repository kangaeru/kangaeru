


\documentclass[12pt,dvipdfmx]{beamer}
\usepackage{graphicx}
\DeclareGraphicsExtensions{.pdf}
\DeclareGraphicsExtensions{.eps}
\graphicspath{{out/}{out/tex/}{out/tex/gpl/}{out/tex/svg/}{out/tex/dot/}}
% \graphicspath{{out/}{out/tex/}{out/pdf/}{out/eps/}{out/tex/gpl/}{out/tex/svg/}{out/pdf/dot/}{out/pdf/gpl/}{out/pdf/img/}{out/pdf/odg/}{out/pdf/svg/}{out/eps/dot/}{out/eps/gpl/}{out/eps/img/}{out/eps/odg/}{out/eps/svg/}}
\usepackage{listings}
\usepackage{fancybox}
\usepackage{hyperref}
\usepackage{color}

\newcommand{\confname}{{○○(仮称)}}
\newcommand{\firstconfdate}{\ao{2017年5月}}
\newcommand{\firstconfplace}{\ao{東京近郊}}
\newcommand{\firstdeadline}{\ao{○年○月}}
\newcommand{\firstgeneralchair}{\ao{○○}}
\newcommand{\firstprogramchair}{\ao{○○}}
\newcommand{\sizeoc}{\ao{○○}}
\newcommand{\sizepc}{\ao{○○}}

%%%%%%%%%%%%%%%%%%%%%%%%%%%
%%% themes
%%%%%%%%%%%%%%%%%%%%%%%%%%%
\usetheme{Szeged} 
%% no navigation bar
% default boxes Bergen Boadilla Madrid Pittsburgh Rochester
%% tree-like navigation bar
% Antibes JuanLesPins Montpellier
%% toc sidebar
% Berkeley PaloAlto Goettingen Marburg Hannover Berlin Ilmenau Dresden Darmstadt Frankfurt Singapore Szeged
%% Section and Subsection Tables
% Copenhagen Luebeck Malmoe Warsaw

%%%%%%%%%%%%%%%%%%%%%%%%%%%
%%% innerthemes
%%%%%%%%%%%%%%%%%%%%%%%%%%%
% \useinnertheme{circles}	% default circles rectangles rounded inmargin

%%%%%%%%%%%%%%%%%%%%%%%%%%%
%%% outerthemes
%%%%%%%%%%%%%%%%%%%%%%%%%%%
% outertheme
% \useoutertheme{default}	% default infolines miniframes smoothbars sidebar sprit shadow tree smoothtree


%%%%%%%%%%%%%%%%%%%%%%%%%%%
%%% colorthemes
%%%%%%%%%%%%%%%%%%%%%%%%%%%
\usecolortheme{seahorse}
%% special purpose
% default structure sidebartab 
%% complete 
% albatross beetle crane dove fly seagull 
%% inner
% lily orchid rose
%% outer
% whale seahorse dolphin

%%%%%%%%%%%%%%%%%%%%%%%%%%%
%%% fontthemes
%%%%%%%%%%%%%%%%%%%%%%%%%%%
\usefonttheme{serif}  
% default professionalfonts serif structurebold structureitalicserif structuresmallcapsserif

%%%%%%%%%%%%%%%%%%%%%%%%%%%
%%% generally useful beamer settings
%%%%%%%%%%%%%%%%%%%%%%%%%%%
% 
\AtBeginDvi{\special{pdf:tounicode EUC-UCS2}}
% do not show navigation
\setbeamertemplate{navigation symbols}{}
% show page numbers
\setbeamertemplate{footline}[frame number]

%%%%%%%%%%%%%%%%%%%%%%%%%%%
%%% define some colors for convenience
%%%%%%%%%%%%%%%%%%%%%%%%%%%

\newcommand{\mido}[1]{{\color{green}#1}}
\newcommand{\mura}[1]{{\color{purple}#1}}
\newcommand{\ore}[1]{{\color{orange}#1}}
\newcommand{\ao}[1]{{\color{blue}#1}}
\newcommand{\aka}[1]{{\color{red}#1}}

\setbeamercolor{ex}{bg=cyan!20!white}

%%%%%%%%%%%%%%%%%%%%%%%%%%%
%%% how to typset code
%%%%%%%%%%%%%%%%%%%%%%%%%%%

\lstset{language = C,
numbers = left,
numberstyle = {\tiny \emph},
numbersep = 10pt,
breaklines = true,
breakindent = 40pt,
frame = tlRB,
frameround = ffft,
framesep = 3pt,
rulesep = 1pt,
rulecolor = {\color{blue}},
rulesepcolor = {\color{blue}},
flexiblecolumns = true,
keepspaces = true,
basicstyle = \ttfamily\scriptsize,
identifierstyle = ,
commentstyle = ,
stringstyle = ,
showstringspaces = false,
tabsize = 4,
escapechar=\@,
}

\title{新規\{ワークショップ,シンポジウム,\ldots\}\confname の提案}
\institute{}
\author{田浦 $\rightarrow$ 考える会}
\date{}

\AtBeginSection[]
{
\begin{frame}
\frametitle{Contents}
\tableofcontents[currentsection]
\end{frame}
}

\iffalse
\AtBeginSubsection[]
{
\begin{frame}
\frametitle{Contents}
\tableofcontents[currentsection,currentsubsection]
\end{frame}
}
\fi

\begin{document}
\maketitle

%%%%%%%%%%%%%%%%%%%%%%%%%%%%%%%%%% 
\begin{frame}
\frametitle{新会議\confname 輪郭}
\begin{itemize}
\item 名称: ????
\item 第一回開催時期: \firstconfdate 
\item 第一回開催場所: \firstconfplace 2-3日
\item 第一回論文募集締め切り時期: \firstdeadline ?
\item 第一回general chair: \firstgeneralchair
\item 第一回program chair: \firstprogramchair
\item 実行委員会規模: 約\sizeoc 
\item プログラム委員会規模: 約\sizepc
\item 論文形式: \ao{査読あり, 予稿集発行せず}
\item 採否決定形式: \ao{ACSIなみ (EasyChair + F2F)?}
\item 論文・発表言語: \ao{日英両方}
\item 学会誌との連携的な話: \ao{しない?}
\end{itemize}
\end{frame}

%%%%%%%%%%%%%%%%%%%%%%%%%%%%%%%%%% 
\begin{frame}
\frametitle{新会議\confname 輪郭: プログラム内容}
\begin{itemize}
\item 通常論文
\item ショート論文(?)
\item 既発表枠
\item 招待講演,基調講演
\item チュートリアル(?)
\item ポスター(?)
\item 企業展示(?)
\end{itemize}

\begin{center}
\ao{\huge だがもっと魅力がほしい!}
\end{center}

\end{frame}

%%%%%%%%%%%%%%%%%%%%%%%%%%%%%%%%%% 
\begin{frame}
\frametitle{開催時期と場所に関して}

\begin{itemize}
\item 一度動いてもらったHPCSにもう一度というのは\ldots
\item 本当に開催時期が重なっているのは避ける
\item 「両方参加可能」を保てばお互いの客の入りに深刻な影響はないと期待
\item HPCSの前後に東京で,というのはありではないか
\item むしろ開催メンバーの負荷スパイクが問題
  特にHPC研究会まわりで,人が重ならないようにできるかという問題?
\end{itemize}
\end{frame}

%%%%%%%%%%%%%%%%%%%%%%%%%%%%%%%%%% 
\begin{frame}
\frametitle{魅力向上案が大事}

\begin{itemize}
\item ACSIをさらに日和って日本語OKにしただけ,とまとめられないためにも
\item 分野の活性化(特に,脱高齢化)のためにも
\end{itemize}

\end{frame}

%%%%%%%%%%%%%%%%%%%%%%%%%%%%%%%%%% 
\begin{frame}
\frametitle{魅力向上案に関連した意見}

\begin{itemize}
\item (研究成果の発信がネットでできる今)「教育的価値」が大事

\item SACSIS/ACSIではチュートリアルの評判高い

\item 巷ではPokemon GOとともに勉強会流行り

\item 賞をいっぱい出す (隣人並みに)
  \begin{itemize}
  \item いろんなベクトルの賞
  \item CS領域奨励賞化する
  \item ちなみにCS領域会議でもどんどん出してと言われた
  \end{itemize}

\item 学生に参加しやすい(例: 無料)

\item 業界の高齢化避けたい

\item Gender Balanceの是正
\end{itemize}
\end{frame}

%%%%%%%%%%%%%%%%%%%%%%%%%%%%%%%%%% 
\begin{frame}
\frametitle{魅力向上関連この間の議論}

\begin{itemize}
\item 学生無料,格安案
\item 若い人枠(学生,修士,学部生)
\item 勉強会セッション的な何か --- 一般化: 「普通の論文以外のセッション」
\item 幅広く、研究者としての人生とか、人生・生活とか
若い人たちが気になることを聞く場、あるいはこうした若い人たちの考えを発信する場
\end{itemize}
\end{frame}


%%%%%%%%%%%%%%%%%%%%%%%%%%%%%%%%%% 
\begin{frame}
\frametitle{その後ふと頭をよぎっていること}

\begin{itemize}
\item 学生のPC体験: 
  \begin{itemize}
  \item 査読とプログラム会議を体験してもらい,どんな議論で論文の採否が決まっているのかを見てもらう(ACSI/SACSISクラスではそんなに意味がない?)
  \item ヒント: NFSの予算審査に若い人を入れるという試みがあったそうです.
  \end{itemize}

\item 論文ではない発表形態全般(コード,データ,etc.)

\item arxivセッション

\item 学会中chat的なもの(個人的には嫌いですが,
  それでフィードバックを得られるなら\ldots)
\end{itemize}
\end{frame}



\end{document}



