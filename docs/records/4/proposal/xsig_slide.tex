


\documentclass[12pt,dvipdfmx]{beamer}
\usepackage{graphicx}
\DeclareGraphicsExtensions{.pdf}
\DeclareGraphicsExtensions{.eps}
\graphicspath{{out/}{out/tex/}{out/tex/gpl/}{out/tex/svg/}{out/tex/dot/}}
% \graphicspath{{out/}{out/tex/}{out/pdf/}{out/eps/}{out/tex/gpl/}{out/tex/svg/}{out/pdf/dot/}{out/pdf/gpl/}{out/pdf/img/}{out/pdf/odg/}{out/pdf/svg/}{out/eps/dot/}{out/eps/gpl/}{out/eps/img/}{out/eps/odg/}{out/eps/svg/}}
\usepackage{listings}
\usepackage{fancybox}
\usepackage{hyperref}
\usepackage{color}

\newcommand{\confname}{{xSIG}}
\newcommand{\firstconfdate}{\ao{2017年5月}}
\newcommand{\firstconfplace}{\ao{東京近郊}}
\newcommand{\firstdeadline}{\ao{2017年1月}}
\newcommand{\firstgeneralchair}{\ao{田浦(東大)}}
\newcommand{\firstprogramchair}{\ao{五島(NII)}}
\newcommand{\sizeoc}{\ao{○○}}
\newcommand{\sizepc}{\ao{○○}}

%%%%%%%%%%%%%%%%%%%%%%%%%%%
%%% themes
%%%%%%%%%%%%%%%%%%%%%%%%%%%
\usetheme{Szeged} 
%% no navigation bar
% default boxes Bergen Boadilla Madrid Pittsburgh Rochester
%% tree-like navigation bar
% Antibes JuanLesPins Montpellier
%% toc sidebar
% Berkeley PaloAlto Goettingen Marburg Hannover Berlin Ilmenau Dresden Darmstadt Frankfurt Singapore Szeged
%% Section and Subsection Tables
% Copenhagen Luebeck Malmoe Warsaw

%%%%%%%%%%%%%%%%%%%%%%%%%%%
%%% innerthemes
%%%%%%%%%%%%%%%%%%%%%%%%%%%
% \useinnertheme{circles}	% default circles rectangles rounded inmargin

%%%%%%%%%%%%%%%%%%%%%%%%%%%
%%% outerthemes
%%%%%%%%%%%%%%%%%%%%%%%%%%%
% outertheme
% \useoutertheme{default}	% default infolines miniframes smoothbars sidebar sprit shadow tree smoothtree


%%%%%%%%%%%%%%%%%%%%%%%%%%%
%%% colorthemes
%%%%%%%%%%%%%%%%%%%%%%%%%%%
\usecolortheme{seahorse}
%% special purpose
% default structure sidebartab 
%% complete 
% albatross beetle crane dove fly seagull 
%% inner
% lily orchid rose
%% outer
% whale seahorse dolphin

%%%%%%%%%%%%%%%%%%%%%%%%%%%
%%% fontthemes
%%%%%%%%%%%%%%%%%%%%%%%%%%%
\usefonttheme{serif}  
% default professionalfonts serif structurebold structureitalicserif structuresmallcapsserif

%%%%%%%%%%%%%%%%%%%%%%%%%%%
%%% generally useful beamer settings
%%%%%%%%%%%%%%%%%%%%%%%%%%%
% 
\AtBeginDvi{\special{pdf:tounicode EUC-UCS2}}
% do not show navigation
\setbeamertemplate{navigation symbols}{}
% show page numbers
\setbeamertemplate{footline}[frame number]

%%%%%%%%%%%%%%%%%%%%%%%%%%%
%%% define some colors for convenience
%%%%%%%%%%%%%%%%%%%%%%%%%%%

\newcommand{\mido}[1]{{\color{green}#1}}
\newcommand{\mura}[1]{{\color{purple}#1}}
\newcommand{\ore}[1]{{\color{orange}#1}}
\newcommand{\ao}[1]{{\color{blue}#1}}
\newcommand{\aka}[1]{{\color{red}#1}}

\setbeamercolor{ex}{bg=cyan!20!white}

%%%%%%%%%%%%%%%%%%%%%%%%%%%
%%% how to typset code
%%%%%%%%%%%%%%%%%%%%%%%%%%%

\lstset{language = C,
numbers = left,
numberstyle = {\tiny \emph},
numbersep = 10pt,
breaklines = true,
breakindent = 40pt,
frame = tlRB,
frameround = ffft,
framesep = 3pt,
rulesep = 1pt,
rulecolor = {\color{blue}},
rulesepcolor = {\color{blue}},
flexiblecolumns = true,
keepspaces = true,
basicstyle = \ttfamily\scriptsize,
identifierstyle = ,
commentstyle = ,
stringstyle = ,
showstringspaces = false,
tabsize = 4,
escapechar=\@,
}

\title{新規ワークショップ \\
\confname の提案と参加のお願い}
\institute{}
\author{研究分野間の新たな連携に関する検討会 \\
(通称: 考える会)}
\date{}

\AtBeginSection[]
{
\begin{frame}
\frametitle{Contents}
\tableofcontents[currentsection]
\end{frame}
}

\iffalse
\AtBeginSubsection[]
{
\begin{frame}
\frametitle{Contents}
\tableofcontents[currentsection,currentsubsection]
\end{frame}
}
\fi

\begin{document}
\maketitle

%%%%%%%%%%%%%%%%%%%%%%%%%%%%%%%%%% 
\begin{frame}
\frametitle{名称\confname のこころ}
\begin{itemize}
\item xSIG: \emph{\ao{cross}}-disciplinary workshop on computing \ao{S}ystems, \ao{I}nfrastructures, and programmin\ao{G} 
\item コンピュータ,システム,基盤,プログラミングに関する分野横断的ワークショップ
\item と同時に, 複数の SIG (研究会) の連携であることを表したい
\end{itemize}
\end{frame}

%%%%%%%%%%%%%%%%%%%%%%%%%%%%%%%%%% 
\begin{frame}
\frametitle{コンセプト}
\begin{itemize}
\item \confname 経由で\ao{国際会議への投稿を妨げない}設計 (ACSI同様)
\item \ao{エントリレベル}含めた若い学生の発表を促進.
その上で,査読,研究会横断など研究会にはない付加価値をつける
\item $\Rightarrow$ ACSI のように英語限定とはせず,
  \begin{itemize}
  \item 国際学会への再投稿を考えている人
  \item エントリとして利用したい人
  \end{itemize}
\ao{両方にメリット}のある設計
\end{itemize}

\end{frame}


%%%%%%%%%%%%%%%%%%%%%%%%%%%%%%%%%% 
\begin{frame}
\frametitle{立て付け}
\begin{itemize}
\item 名称: \confname
\item 第一回開催時期: \firstconfdate 
\item 第一回開催場所: \firstconfplace 2-3日
\item 第一回論文募集締め切り時期: \firstdeadline ?
\item 第一回general chair: \firstgeneralchair
\item 第一回program chair: \firstprogramchair
\item 論文形式: \ao{査読あり, 予稿集発行せず}
\item 採否決定形式: \ao{ACSIなみ (EasyChair + F2F)?}
\item 論文・発表言語: \ao{日英両方}
\item 学会誌との連携的な話: \ao{しない?}
\end{itemize}
\end{frame}

%%%%%%%%%%%%%%%%%%%%%%%%%%%%%%%%%% 
\begin{frame}
\frametitle{魅力向上策(検討中)}
\begin{itemize}
\item 一部の有望論文に\ao{メンター}をつけ,国際学会への投稿推奨
\item 一定数, \ao{学生枠, 修士以下枠, 学部生枠}をもうけ,学生の投稿を推奨
\item 最優秀論文賞的な賞の他に, \ao{奨励賞を多数,色々な尺度}で出す
(アイデア,実装の努力や完成度,説明が素晴らしい,英語が素晴らしい,etc.)
\item (少なくとも初年度は)学生の参加を無料とする.
または,遠隔からの発表者に旅費援助 ?
\item \ao{ArXiv}との連携? 登録済みの英語論文を無条件に発表許可?
\item 卒論,修論\ao{丸投げ}制度?
\end{itemize}
\end{frame}


%%%%%%%%%%%%%%%%%%%%%%%%%%%%%%%%%% 
\begin{frame}
\frametitle{魅力向上策(検討中)}
\begin{itemize}
\item \ao{チュートリアル的}な,広い聴衆向けに基礎を紹介する系のプロ
グラムを充実させる,ないし気楽に多数実行
\begin{itemize}
\item やってほしいチュートリアルの\ao{募集}
\item \ao{成果ソフトウェアのハンズオン}チュートリアル的なもの 
  (AICS のソフトウェアなど )
\end{itemize}
\item いわゆる論文ではなく,他の研究者に貢献する
\ao{プログラムコードやデータ}の公開を目的とした発表など, 多様な発表・貢献形態
\end{itemize}
\end{frame}

%%%%%%%%%%%%%%%%%%%%%%%%%%%%%%%%%% 
\begin{frame}
\frametitle{目指すところ}
研究会ごとの細分化ではなく
基盤,システム系分野が一同に介して交流できる学会を目指しています


{\huge よろしくおねがいします}

\end{frame}


%%%%%%%%%%%%%%%%%%%%%%%%%%%%%%%%%% 
\begin{frame}
\frametitle{HPCSとの開催時期の重複に関して}
\begin{itemize}
\item 過去にACSIのために動いてもらったという経緯も有り,心苦しい
\item 本当に開催時期が重なっているのは避けるというのが大前提
\item 「両方参加可能」を保てばお互いの客の入りに深刻な影響はないと期待
\item HPCSの前後に東京で,というのはありではないか
\end{itemize}
\end{frame}

%%%%%%%%%%%%%%%%%%%%%%%%%%%%%%%%%% 
\begin{frame}
\frametitle{考える会「国際会議化」に関しての議論}

\begin{itemize}
\item 初回,色々な立場の人にポジショントークをしてもらい,
  中にはもちろんフル国際会議化を推す意見もあった
\item アンケートでも,フル国際会議化という意見は多い
\item 一方,フル国際会議は,
  \begin{itemize}
  \item 分野が広いほど,「需要の明確化」「立ち位置の確保」が難しくなり,
    成功の青写真が描きにくくなる
  \item 各分野,トップ会議のラインアップは確立されており,
    かつすでに過密気味
  \end{itemize}
\item ここでの第一義的な目的は,
  日本の「システム〜高性能応用」研究を縮小させないため,
  次の世代を盛り上げるための,分野間連携
\item フル国際会議はもう少し絞った分野で,分野関連系はエントリ重視で,
  というすみ分け
\end{itemize}
\end{frame}


\end{document}



